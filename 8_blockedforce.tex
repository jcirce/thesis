\chapter{Measuring Blocked Force}
\label{chapter:blockedforce}

\section{Overview}
Characterization of the blocked force for a soft pneumatic actuator is essential to understanding the behavior and provides insight into possible applications for the robot. Blocked force results from the bending moments inside the actuator, which has nowhere to go. During unrestricted bending, the moment from the input pressure is equal to the moment from axial strain along the length of the actuator. If we prohibit axial strain, the internal pressure is resolved by an external or blocked force, $F$. Fig.~\ref{fig:justblockedforce} contains Fig.~\ref{fig:modelall}C. Still, in this chapter, we finally define $F$, the blocked force bending moment generated in response to input pressure, $P$, if we restrict axial strain ($\lambda_{\varphi,\tau}$ and $\lambda_\beta$). 

\begin{figure}[ht]
    \centering
     \includegraphics[width=3 in]{images8/justblockedforce.jpg}
    \caption{Defining blocked force, $F$, in relation to the analytical model.}
    \label{fig:justblockedforce}
\end{figure}

The standard fiber-reinforced actuator's blocked force can be measured by placing the load cell underneath the end of the actuator, and with pressure, the actuator begins bending, but since its motion is restricted by the load cell, the force reading on the load cell is the blocked force of the actuator. Additional materials are placed on the actuator for a \emph{more accurate} reading to ensure no axial strain. This method of measuring and characterizing the blocked force is acceptable for the standard actuator. However, the circular actuator's shape and bending behavior create several challenges when measuring the blocked force. 

\section{Iterations on Orientation}

The load cells we purchased measure a force in a single direction (a 1-axis load cell). To calibrate the sensor, we used objects of known weight (often the clamps used for holding the molds together during fabrication). First we took readings with no load on the load cell to know the initial value, then we added the object of known weight and used a string to hang it from the load cell. We took several readings at different weights to generate a linear map of the digital reading from the load cell's amplifier to the force applied from the object using gravity. Calibrating the load cell in this way meant that the load cell must remain pointing in the direction of gravity (towards the lab bench). 

\subsection{Pushing Down on the Load Cell}

For the first attempt at using the load cell to measure the blocked force at the end of the actuator, we fixed the load cell so that it would measure a force normal to the table (in the same direction as gravity). We wanted the flat side of the actuator (the side with the fiberglass fabric) to just touch the load cell when the actuator before we added pressure. We would be able to calibrate out the weight of the actuator itself (or so we thought). Fig. \ref{fig:pushingdown} contains photographs taken while we measured blocked force. Because each actuator had a slightly different length from fabrication inconsistencies, the ideal angle of the end with the barb was slightly different for each actuator. We tried holding the barb end of the actuator at 45 degrees using a triangle, the free end was able to connect with the load cell, but not perfectly in parallel. We also tried holding the barb end perpendicular to the table (in parallel with the direction of the force on the load cell), but faced the same problem. 

\begin{figure}[ht]
    \centering
     \includegraphics[width=6 in]{images8/pushingdown.jpg}
    \caption{Pushing down into the load cell setup to measure blocked force. A. SS40 actuator held at 45$^\circ$ using the triangle. B. DS20 actuator also held at 45$^\circ$. C. The same DS20 actuator at a higher pressure. D. SS40 actuator held perpendicular to the table. E. DS20 actuator held in the same configuration as D.}
    \label{fig:pushingdown}
\end{figure}

We were able to capture force readings using this method, but several factors led to ultimately characterizing the blocked force using a different method. First, calibrating out the weight of the actuator from the force reading was non-trivial because as pressure is added to the actuator, the stress from axial strain would lift the actuator up away from the load cell, decreasing the force reading. Also, as we increased pressure, the end of the actuator would slide along the face of the load cell, changing the orientation of the force from pressure on the end of the actuator, changing the direction of the blocked force. When the end of the actuator was significantly misaligned from the load cell (Fig. \ref{fig:pushingdown}C), since the load cell only captures force in one direction, the reading would not capture the entire blocked force. 

We attempted a few methods of fixing the end of the actuator to the load cell, we used electric tape, Kevlar thread, and even a 3D printed jig. Despite helping to maintain the contact between the end of the actuator and the load cell, since the actuator wanted to pull away from the load cell, these additions would also cause a vertical force on the load cell, further complicating the reading. 

\subsection{Pulling Up on the Load Cell}

To ensure the entire blocked force on the end of the circular actuator could be read by the load cell, we used Kevlar thread to resolve the force into a single axis, the axis of the load cell. By tying the thread around the free end of the actuator, and tying the other end to the load cell, we could block some axial deformation and record the force. To align the one side of the thread with the axis of the load cell, we used a small bearing mounted to a 3D printed block. This block would be fixed a suitable height above the load cell. The flange on the bearing helped keep the thread aligned with the actuator and the load cell. This orientation of the actuator suffered from the same problems as measuring the bending angle against gravity. The actuator was free to bend and twist out of the plane the bearing and the load cell were in. The thread then had the freedom to move in 3 axes, which was not ideal as sometimes the thread would fall off the bearing. Fig \ref{fig:verticalstring} contains photographs of this testing setup. We determined the optimal length of the string such that no force would be read on the load cell at 0.0 psi. 

\begin{figure}[ht]
    \centering
     \includegraphics[width=4.5 in]{images8/verticalstring.jpg}
    \caption{Photographs of the circular actuator connected to the load cell over a bearing acting as a pulley to resolve the multi-directional force on the end of the actuator into a single axis for the load cell reading. The orange dashed lines are parallel to the inextensible thread (added for emphasis).}
    \label{fig:verticalstring}
\end{figure}

While this testing may not be considered \emph{true} blocked force testing because the actuator undergoes some axial strain, even if we were able to keep the actuator in a circular shape, the circumferential and radial expansion would still induce axial stress, and that stress would still pull away from the load cell, altering the force reading. Modeling the blocked force as purely a function of cross section and pressure, not including any non-axial strain would lead one to believe there is a linear relationship between pressure and blocked force. While this may be true for standard soft pneumatic actuators, the circular actuator's initial bending angle and shape is unique and because we do not need strain limiting materials to limit circumferential and radial strain to achieve the bending angle, the blocked force reading will always be a function of the stress in the silicone. 

Using an inextensible thread that was free to rotate around the bearing did not fully block the actuator from unbending. However, it did restrict some axial deformation, inducing a non-constant curvature shape of the fiberglass layer. With increasing pressure, the angle of the thread with respect to the the axis of the load cell decreased until both halves of the thread were parallel and the thread would lift off the pulley. Fig \ref{fig:verticalstringangle} contains samples from various pressures of a DS20 actuator blocked with an inextensible thread attached to the load cell over a bearing acting as a pulley. Note how the shape of the actuator becomes less circular with increasing pressure and the angle between the two pieces of thread reduces until the thread lifts off the pulley. 

\begin{figure}[ht]
    \centering
     \includegraphics[width=5 in]{images8/verticalstringangle.jpg}
    \caption{Photographs of a DS20 actuator at pressures ranging between 0.2 and 15.2 psi. The bending angle of the actuator is blocked by an inextensible thread attached to a load cell over a pulley. The orange line indicates the angle of the thread connected to the free end of the actuator.}
    \label{fig:verticalstringangle}
\end{figure}

\clearpage
\subsection{Pulling on the Load Cell while Horizontal}

Similar to the vertical bending angle experiments, when the actuator is fixed vertically (perpendicular to the table), gravity does influence the initial bending angle: the weight of itself causes deformation without input pressure, especially for the softer silicones. Additionally, it was difficult to measure the length of the string while the end of the actuator was floating. We were satisfied with using the thread to resolve the multi-directional force into a single axis for the load cell compared to the end of the actuator directly pressing on the load cell. To solve this, we rotated the actuator and added a piece of aluminum for the actuator to rest on. Having the actuator flat on the aluminum forced the actuator to stay in one place, removing any third axis twisting or bending. Just as with the horizontal bending angle experiments, we removed the effect of gravity on the bending, but now friction would inhibit the actuator's motion. Fortunately, during blocked force testing, the actuator's \emph{bending angle} does not decrease significantly, so we tolerated the influence of friction. 

In this testing orientation, the inextensible thread still runs over the pulley, but we could add more pressure to the actuator before the string would fall off the pulley compared to the vertical experiments. In this orientation, the length of the thread was measured such that the actuator maintained its initial bending angle and circular shape, and the thread was perpendicular to the load cell. With increasing pressure, as the actuator begins to lose its circular shape, the thread rotates away from the pulley. As shown in Fig \ref{fig:horizontalblockedforce}A, with pressure ($P > 0$), the angle of the string increases relative to the load cell. Fig \ref{fig:horizontalblockedforce}B contains photographs of a DS20 actuator, highlighting the angle of the thread. 

\begin{figure}[ht]
    \centering
     \includegraphics[width=5 in]{images8/horizontalblockedforce.jpg}
    \caption{Measuring blocked force with the actuator resting on a horizontal surface. A. Drawing of how the shape of the actuator changes with pressure, showing how the angle of the thread changes. B. Photographs of a DS20 actuator with an orange line showing the angle of the string with increasing pressure.}
    \label{fig:horizontalblockedforce}
\end{figure}