\appendix

\chapter{PCL Inserts}
\label{appendix:pcl}

PCL (polycaprolactone) is a plastic with a low melting temperature and is moldable in warm water. Using the same silicone mold made for the wax inserts, we used warm water to melt the PCL pellets into blobs, which we pressed into the silicone mold. The DS20 mold was far too flexible to shape the PCL accurately. Also, as the pieces of PCL cooled and solidified, they would not fully bond with the other pieces, leaving air gaps in the insert. After tuning the timing between the cooling down and adding more PCL pieces, we successfully created an insert and cast a DS20 actuator around the PCL insert. Unfortunately, the silicone walls were uneven, and the PCL insert had the same single-use problem as the wax inserts. Fig. \ref{fig:pclinsert} contains photographs of experimenting with PCL as an insert and the DS20 actuator we cast using the PCL insert.

\begin{figure}[!ht]
    \centering
    \includegraphics[width=6 in]{images4/pclinsert.jpg}
    \caption{Experimentation with PCL inserts. A. Silicone mold. B. An attempt. C. Melting PCL in warm water. D. PCL insert inside PLA mold. E. Casted around PCL insert. F. Removed from mold. G. Extra silicone removed. H. Final actuator.}
    \label{fig:pclinsert}
\end{figure}

\chapter{Non-linear Pressure From Digital Potentiometer}
\label{appendix:bitmap}

\begin{figure}[!ht]
    \centering
    \includegraphics[width=6.5 in]{images6/bitpressuremap.jpg}
    \caption{Relationship between the range set on the pressure regulator, the command sent to the digital potentiometer and the output pressure in the actuator for four pressure ranges.}
\end{figure}

\chapter{Bending Angle Orientations}
\label{appendix:ao}

\begin{figure}[!ht]
    \centering
     \includegraphics[width=6 in]{images7/firstdataset.jpg}
    \caption{Photographs taken when the end of the circular actuator is fixed to the table. Pressure increments are $1.0~\pm~0.1$ from 2.0 to 12.0 psi.}
    \label{fig:firstdataset}
\end{figure}

\begin{figure}[!ht]
    \centering
     \includegraphics[width=6 in]{images7/ss45fallsover.jpg}
    \caption{Photographs of a SS45 actuator at 2.0, 9.2, and 12.8 psi.}
    \label{fig:ss45fallsover}
\end{figure}

\clearpage

\begin{figure}[!ht]
    \centering
     \includegraphics[width=6 in]{images7/ds20h.jpg}
    \caption{Photographs taken of a DS20 actuator at various psi pressures when placed horizontally on the lab bench.}
    \label{fig:ds20h}
\end{figure}

\clearpage
\begin{figure}[!ht]
    \centering
     \includegraphics[width=5.5 in]{images7/withgravity.jpg}
    \caption{Photographs taken of a DS20 actuator when held vertically bending with gravity.}
    \label{fig:withgravity}
\end{figure}

\begin{figure}[!ht]
    \centering
     \includegraphics[width=5.5 in]{images7/d30withgravity.jpg}
    \caption{Photographs taken of a DS30 actuator at various psi pressures when held vertically and allowed to unbend with gravity.}
    \label{fig:d30withgravity}
\end{figure}

\clearpage

\begin{figure}[!ht]
    \centering
     \includegraphics[width=6.5 in]{images7/ds20sver.jpg}
    \caption{Photographs taken of a DS20 actuator vertically but offset from the table with standoffs. The left group was increasing the pressure, and the right group was from decreasing the pressure.}
    \label{fig:ds20sver}
\end{figure}

\clearpage

\begin{figure}[!ht]
    \centering
     \includegraphics[width=6.5 in]{images7/d20againstgravity.jpg}
    \caption{Photographs of a DS20 actuator across its entire bending range of motion with pressures labeled in kPa. The left group was increasing pressure increments, and the right group was from decreasing pressure.}
    \label{fig:d20againstgravity}
\end{figure}

\clearpage
\chapter{DS20 Bending Angle}
\label{appendix:d20all}

\begin{figure}[ht]
    \centering
     \includegraphics[width=6.5 in]{images9/d20samples.jpg}
    \caption{Photographs used to measure bending angle of a DS20 circular actuator. The pressure is in kPa. The left group is increasing pressure, the right group is decreasing pressure.}
    \label{fig:d20samples}
\end{figure}

\chapter{DS30 Bending Angle}
\label{appendix:d30all}

\begin{figure}[ht]
    \centering
     \includegraphics[width=6.5 in]{images9/d30samples.jpg}
    \caption{Photographs of a DS30 actuator across its entire bending range from 7-168~kPa. The left group is increasing pressure, the right group is decreasing pressure.}
    \label{fig:d30samples}
\end{figure}

\begin{figure}[ht]
    \centering
     \includegraphics[width=6.5 in]{images9/seahorseeffect.jpg}
    \caption{Photographs of a DS30 actuator experiencing the ``Seahorse Effect'', non-uniform circumferential expansion caused more than one concavity in a single circular actuator. The left group is increasing pressure, the right group is decreasing pressure.}
    \label{fig:seahorseeffect}
\end{figure}

\chapter{SS40 Bending Angle}
\label{appendix:s40all}

\begin{figure}[ht]
    \centering
     \includegraphics[width=6.5 in]{images9/ss40samples.jpg}
    \caption{Photographs of a SS40 actuator with input pressures from 21-256~kPa. The left group is increasing pressure, the right group is decreasing pressure.}
    \label{fig:ss40samples}
\end{figure}

\chapter{SS50 Bending Angle}
\label{appendix:s50all}
\begin{figure}[ht]
    \centering
     \includegraphics[width=6.5 in]{images9/ss50samples.jpg}
    \caption{Photographs of a SS50 actuator with input pressures from 21-256~kPa. The left group is increasing pressure, the right group is decreasing pressure.}
    \label{fig:ss50samples}
\end{figure}

\chapter{Pulling up on Load Cell}
\label{appendix:pullinguploadcell}

\begin{figure}[ht]
    \centering
     \includegraphics[width=4.5 in]{images8/verticalstringangle.jpg}
    \caption{Photographs of a DS20 actuator at pressures ranging between 0.2 and 15.2 psi. The orange line indicates the angle of the thread connecting the free end of the actuator to the load cell.}
    \label{fig:verticalstringangle}
\end{figure}

\chapter{Blocked Force Photographs}
\label{appendix:bf}

\begin{figure}[ht]
    \centering
     \includegraphics[width=5 in]{images10/d20eccentricity.pdf}
    \caption{Photographs of a DS20 actuator at pressures ranging between 0 and 105 kPa during blocked force testing.}
    % \label{fig:verticalstringangle}
\end{figure}

\begin{figure}[ht]
    \centering
     \includegraphics[width=5 in]{images10/d30eccentricity.pdf}
    \caption{Photographs of a DS30 actuator at pressures ranging between 0 and 112 kPa during blocked force testing.}
    % \label{fig:verticalstringangle}
\end{figure}

\begin{figure}[ht]
    \centering
     \includegraphics[width=5 in]{images10/s40eccentricity.pdf}
    \caption{Photographs of a SS40 actuator at pressures ranging between 0 and 210 kPa during blocked force testing.}
    % \label{fig:verticalstringangle}
\end{figure}

\begin{figure}[ht]
    \centering
     \includegraphics[width=5 in]{images10/s50eccentricity.pdf}
    \caption{Photographs of a SS50 actuator at pressures ranging between 0 and 210 kPa during blocked force testing.}
    % \label{fig:verticalstringangle}
\end{figure}