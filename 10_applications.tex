\chapter{Applications}

\section*{Overview}
This chapter details the force-based applications of the circular actuator. Like any soft robot, the circular actuator's silicone exterior conforms to the object it holds and can grab and hold objects without causing damage. This chapter includes applications for a single circular actuator, but controlling more than one would allow for more applications. For a single actuator, first, we documented picking up random objects found in the lab. For round objects of various diameters, we placed the circular actuator inside the object to pick it up from the inside, and for larger objects, we used the negative $\psi$ bending region to grasp these objects from the outside. While picking these objects helps us understand the actuator's capabilities and to further explore blocked force when the actuator develops an eccentricity, we designed an experiment to measure the maximum weight that actuators of each material at varying input pressures could hold when placed inside an object.

\section{Three Random Objects}
Demonstrating its viability as a soft robot, the circular actuators can interact with small objects of any shape without causing damage to the object. For a hollow 3D-printed dodecahedron, the circular actuator, once placed inside, conforms to the object's walls when pressurized. With increasing pressure, the actuator's curvature and circumferential expansion conform to the object, and with pressures inducing a negative $\psi$, the actuator can hold the object above itself. Fig. \ref{fig:3dprintedonend} contains photographs of using a DS20 circular actuator to pick up a 3D-printed hollow dodecahedron placed on the end. 

\begin{figure}[!ht]
    \centering
    \includegraphics[width=6.5 in]{images10/3dprintedonend.jpg}
    \caption{Photographs of a DS20 circular actuator with increasing pressure picking up a hollow 3D-printed dodecahedron.}
    \label{fig:3dprintedonend}
\end{figure}

For objects smaller than the initial bending radius of the circular actuator, simply placing the unpressurized actuator on one side of the object is enough to grab the object. For a box of gloves, a DS20 actuator can self-align with the object with increasing pressure. Fig. \ref{fig:boxofgloves} contains photographs of a DS20 actuator picking up a box of gloves. For this object, we stepped from 0 kPa to a pressure high enough to induce a negative $\psi$ based on the size of the box. The first photograph is of the actuator resting on the box. The second and third photographs are of the actuator mid-pressurization flipping its orientation. The fourth is once the actuator reaches the desired input pressure. Holding the actuator from the center can easily pick up the box. Generating sufficient $\psi$ to conform around the object from rest with increasing pressure without human input showcases the versatility of the circular actuator. 

\begin{figure}[!ht]
    \centering
     \includegraphics[width=6.5 in]{images10/boxofgloves.jpg}
    \caption{Photographs of a DS20 circular actuator with increasing pressure picking up a box of gloves.}
    \label{fig:boxofgloves}
\end{figure}

The circular actuator can more easily conform to the object's radius for large round objects. With increasing pressure, a circular actuator will reach the negative $\psi$ corresponding to the object's size. Once the object restricts the bending angle, increasing the pressure further generates a force on both ends of the actuator corresponding to how much the object is restricting the bending angle. For a volleyball, as shown in Fig. \ref{fig:aroundvolleyball}, a DS20 actuator, when held against the volleyball, can conform to the size of the ball and, with enough pressure, generate enough force on each end to be able to pick up the ball when held in the center. Similar to the box of gloves, the second photograph is mid-pressurization, before the volleyball restricts the actuator's bending.

\begin{figure}[!ht]
    \centering
     \includegraphics[width=6.5 in]{images10/aroundvolleyball.jpg}
    \caption{Photographs of a DS20 circular actuator with increasing pressure picking up a volleyball.}
    \label{fig:aroundvolleyball}
\end{figure}

\section{Round Objects}

The blocked force experiments provided insight into the force generated on one side of a circular actuator when restricted at $\psi_0$. To estimate the force generated with increasing pressure when blocking the actuator at a negative $\psi$ requires more information. First, we know the actuator requires an input pressure to generate the negative bending angle corresponding to the diameter of the round object. We know the force generated on each end of the actuator is proportional to the moment from pressure on the end cap, provided the object restricts axial deformation; increasing pressure generates the blocked force, $F$. However, since the object is not restricting circumferential deformation for the semi-circular region of silicone, the circumferential expansion also generates additional axial stress, which further helps the actuator grab the object. While analytically modeling the blocked force for the circular actuator when restricted at a given $\psi$ is possible, we did not include it in this work. Additionally, we did not calculate the required force the actuator would need to hold round objects, knowing their weight and the friction between the contact surface of the painted DS10 silicone on the exterior of the actuator and the object. 

We used a DS30 actuator and a round container with a 15~cm diameter to show the possibilities of the circular actuators. When placed inside the container, we pressurized the DS30 actuator to 110~kPa, which generated a $\psi$ so that the container walls blocked the actuator. We added weights (clamps and fasteners) to the inside of the container to showcase the actuator's strength. At 110~kPa, the DS30 actuator could hold 380~g inside the container. To hold the container from the outside, we used an input pressure of 165~kPa. We found that the DS30 actuator could hold 420~g inside the container at this pressure. The DS30 actuator could pick up the volleyball (270~g, outer diameter of 20~cm) at 150 kPa. Fig. \ref{fig:ds30roundobjects} contains photographs of a DS30 actuator holding the container and the volleyball. We used Kevlar thread placed around the center of the actuator to pick up the objects. 
\\
\begin{figure}[!ht]
    \centering
     \includegraphics[width=6.5 in]{images10/ds30roundobjects.jpg}
    \caption{Photographs of a DS30 circular actuator picking up a container with 15~cm diameter and a volleyball of 20~cm diameter. The input pressures and weight of the objects are labeled.}
    \label{fig:ds30roundobjects}
\end{figure}

While these are not necessarily the maximum weight that the DS30 actuator could pick up for round objects of these diameters, these experiments show some possible applications for circular actuators. To better understand the relationship between blocked $\psi$ (the size of the round object), input pressure, and generated force, analytically modeling how the continued circumferential expansion impacts axial stress is required. 

\section{Picking up a Cup}

Using an object with an internal diameter of 9~cm, smaller than the $\psi_0$ bending radius, we placed circular actuators inside the object. With increasing pressure, we measured the maximum weight actuators of all four materials could hold. This experiment explores how adding internal pressure to the actuator increases the blocked force generated when constrained at a certain eccentricity. For each material, at each pressure increment, we added more weight (using 50~g increments) inside the object until the actuator slipped out of the object when picked up from the center of the actuator. All three actuators (DS20, DS30, and SS40) held 700~g within a pressure range of 0-60~kPa when placed inside the cup. Increasing material stiffness could hold about 50~g more weight for a given input pressure. Fig. \ref{fig:coopercupdata} contains the weight actuators could hold inside the cup for pressures between 20-90~kPa. At pressures higher than 60~kPa, the DS20 actuator buckled in the center and could not maintain contact with the inner walls of the object. The DS30 and SS40 actuators at 80~kPa held 750~g and 800~g, respectively. 
\\
\\
\begin{figure}[!ht]
    \centering
     \includegraphics[width=4 in]{images10/coopercupdata.jpg}
    \caption{Maximum weight inside an object of 9~cm inner diameter actuators of three materials could hold without slipping.}
    \label{fig:coopercupdata}
\end{figure}

For objects with an inner diameter requiring us to compress the actuator so that it can make contact with both walls of the object, the compression of the actuator itself generates a blocking force on the wall of the object. For the stiffness SS50 actuators, at just 20~kPa, the actuator could pick up 1000~g of weight inside the cup, a significant increase compared to the other materials. Fig. \ref{fig:coopercupphotos} contains photographs of actuators of each material holding increasing weight at different pressures inside a cup. 
\\
\begin{figure}[!ht]
    \centering
     \includegraphics[width=6.5 in]{images10/coopercupphotos.jpg}
    \caption{Photographs of circular actuators with internal pressure to pick up a cup with increasing weight.}
    \label{fig:coopercupphotos}
\end{figure}

\clearpage
\section{More on Eccentricity}

As discussed in section \ref{section:eccentricity}, the blocked force generated by circular actuators is a function of input pressure and is related to the eccentricity developed when constraining the actuator. For the cup with a 9~cm inner diameter, the actuators were blocked at an eccentricity of approximately 0.7, a value higher than previous blocked force experiments. Additionally, it is worth noting that for a given eccentricity, the non-uniform curvature and non-uniform axial deformation along the actuator's length is a function of the location of both ends of the actuator along the ellipse. Fig. \ref{fig:comparingeccentricity} contains photographs of the actuator's eccentricity during blocked force testing compared to when placed inside a cup. 

\begin{figure}[!ht]
    \centering
     \includegraphics[width=4.5 in]{images10/comparingeccentricity.jpg}
    \caption{Photographs of circular actuators during blocked force testing compared to applications testing, defining variables for tracking eccentricity. }
    \label{fig:comparingeccentricity}
\end{figure}

For blocked force experiments, we fixed one end of the actuator on the semi-minor axis, and the \emph{free} end exists in the second quadrant (using the orange lines drawn in Fig \ref{fig:comparingeccentricity}). When placed inside the cup, the ends of the actuator are in the second and third quadrants. So, while we can measure the eccentricity formed by completing the ellipse, a single $e$ value does not fully characterize the actuator's shape. To define the actuator's elliptical shape and calculate the changes in curvature over the length of the actuator, provided the semi-major axis of the ellipse is aligned with the horizontal axis, we can define $\gamma$ as the angle from the semi-major axis to one end of the actuator, and $\theta$ as the angle from $\gamma$ to the other end of the actuator. Including $a_e$ and $b_e$, the elliptical shape formed by the actuator is fully defined. Using this information to break away from the constant curvature assumption, we could create a model for blocked force for the circular actuator. 
