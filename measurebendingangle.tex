\chapter{Measuring Bending Angle}
\label{chapter:bendingangle}

\section*{Overview}
This chapter contains how we measured the bending angle of the circular actuators at different pressures, the equipment used, the various orientations we used to photograph the actuator, testing protocols, how we processed the images, and how we calculated the bending angle.

\section{Photographing Bending Angle}

To experimentally measure the bending angle of the circular actuators, we used a camera to take photos of the actuator at each pressure. We entertained using a flexible resistor (SEN-08606, SparkFun), but this would not provide insight into variations in curvature or non-axial expansion along the length of the actuator. A photograph, although requiring post-processing, would allow us to characterize the bending behavior. We used a web camera (C920 Hd Pro, Logitech) to take photographs of the actuator at each pressure.

\section{Iterations on Orientation}

As with everything related to this project of designing and characterizing a new type of soft robotic actuator, discovering how to measure the bending angle at different pressures repeatably took several iterations. We needed to develop a way to hold the actuator in front of the camera so we could measure the bending angle and compare it to the model. Since the fiberglass fabric embedded into the actuator determines the bending angle, the camera must be placed perpendicular to the fabric. We also wanted to capture the actuator's circumferential expansion in the photographs. 

Appendix \ref{appendix:ao} contains the complete process for how we determined the best orientation to photograph bending angle. Samples of the various orientations are contained in Fig \ref{fig:orientationsall} which includes testing vertically where the actuator can make contact with the table (Fig \ref{fig:orientationsall}A), horizontally where friction impacts the behavior (Fig \ref{fig:orientationsall}B), vertically with gravity (Fig \ref{fig:orientationsall}C), vertically but lifted off the table (Fig \ref{fig:orientationsall}D), and finally vertically against gravity (Fig \ref{fig:orientationsall}E). Each of these orientations suffered from external influences creating non-constant curvature in the circular actuator. 

\begin{figure}[!ht]
    \centering
    \includegraphics[width=6.5 in]{images7/orientationsall.pdf}
    \caption{Various orientations initially used to photograph bending angle. Contains photographs of DS20 actuators at 2.0, 6.0, 10.0, and 12.1 psi}
    \label{fig:orientationsall}
\end{figure}

\clearpage
\subsection{Chosen Orientation}

In order to remove the influence of gravity and friction on the bending angle as we chose to photograph the actuator horizontally. If the fiberglass fabric was placed perpendicular to the table, using a camera mounted above the table pointing down, we could capture the bending angle at any pressure. For simplicity and to not require additional hardware, we held the actuator above the table when changing the internal pressure to eliminate the effects of friction. Once the pressure regulator reached a steady state, we placed the actuator on the table before taking the photograph. Fig. \ref{fig:chosenorientation} contains photographs taken using a DS20 actuator in the chosen testing orientation. 

\begin{figure}[!ht]
    \centering
    \includegraphics[width=6 in]{images7/chosenorientation.pdf}
    \caption{DS20 actuators photographed at pressures between 0-14.5 psi.}
    \label{fig:chosenorientation}
\end{figure}

\clearpage
\section{Testing Protocols}

Using the terminal output and keyboard of the computer, the person running the test would see the pressure about to be sent to the actuator and then press the space bar to confirm the pressure should be changed. Once the actuator had reached a steady state at that pressure, pressing the ``m" key would take a photograph, name it with the name of the actuator and the current pressure, and save it to a directory with the name of the test. The script would continue through the list of pressures we wanted to test the actuator at until the peak pressure was reached. After documenting the bending angle at the maximum pressure, the pressure increments would reverse to capture hysteresis in the bending behavior. Each test took 5-10 minutes to take 30-35 photographs, depending on the achievable peak pressure in the actuator.

\section{Calculating Bending Angle}

Now that we have an effective method of photographing the circular actuators, we can use the photographs to calculate the bending angle at each pressure. Starting from the initial bending angle, $psi_0$ and rearranging the arc-length equation, $l_{0} = r_{0}\psi_{0}$, for the bending angle gives $\psi = l_0/r$, where $l_0$ is the length of the inextensible fiberglass fabric, and $r$ is the bending radius of the actuator. Measuring the curvature, $\kappa$, from a photograph of the fiberglass layer is the easiest approach to calculate the bending angle. Knowing $\kappa = 1/r$, we can calculate the bending angle from the curvature using $\psi = l_{0}\kappa$. 

\subsection{Curvature from Three Points}

Given three coordinates, $(x_1,y_1)$, $(x_2,y_2)$, $(x_3,y_3)$, in cartesian space, we can calculate the curvature of the circle formed by those three points using Eq. \ref{eq:CurvatureEq} \cite{ratliff_cartesian_2019}. 
\begin{align} 
    \kappa = \frac{2\cdot\lvert((x_2-x_1)\cdot(y_3-y_2)) - ((y_2-y_1)\cdot(x_3-x_2))\rvert}{\sqrt{[(x_2-x_1)^2+(y_2-y_1)^2] \cdot [(x_3-x_2)^2+(y_3-y_2)^2] \cdot [(x_1-x_3)^2+(y_1-y_3)^2]}} 
    \label{eq:CurvatureEq} 
\end{align}

We began by comparing the model to the experimental results using OnShape CAD software. We manually selected three points along the fiberglass layer and used the built-in functionality to measure the radius of a circle defined by three points. The checkerboard grid has 1~cm squares, which we also measured to convert the radius of the circle to meaningful units. Fig. \ref{fig:onshapecircles} contains two samples of sketches of circles created in OnShape. Note at 9.1 psi, the actuator has a positive $\psi$, and at 12.9 psi, the actuator has a negative $\psi$. Also note that as $\psi$ approaches $0^\circ$, the circle's radius approaches infinity. 

\begin{figure}[ht]
    \centering
     \includegraphics[width=6 in]{images7/onshapecircles.pdf}
    \caption{Two screenshots from OnShape with the 3 points used and the calculated bending radius, $r$ are labeled for a DS20 actuator at 9.1 and 12.9 psi.}
    \label{fig:onshapecircles}
\end{figure}

\section{Using OpenCV to Calculate Bending Angle}
\label{section:opencv}

To provide automation, we wrote a Python script using OpenCV \cite{opencv_library}. For each photograph, the script would detect the fiberglass fabric layer as a contour and use points along the contour to calculate the average curvature of the circular actuator at that pressure. 

While first developing the line-detection script, we wanted to detect both the actuator's outer and inner walls to measure the actuator's circumferential expansion. Additionally, we wanted to develop a method to measure the curvature over the length of the actuator for experimental setups where gravity or friction caused the actuator to have non-constant curvature. The first scripts utilized the black and blue checkerboard pattern to remove the distortion from the camera. Limitations of this method included not necessarily knowing which coordinates on the contour line were from the inside, outside, or end caps of the actuator. The curvature calculation requires three $x$ and $y$ coordinates from the fiberglass fabric layer. However, we needed a new method to automatically detect the correct contour and measure the bending angle. 

Measuring circumferential expansion using the photographs would provide some insight into the actuator's behavior, but ultimately, since the three strains are interconnected, measuring each strain separately provides less insight than we had hoped. After we concluded that measuring only the curvature of the fiberglass layer would provide enough information to compare to the model, we decided to use a dry-erase marker to draw a line on the silicone. We used a blue or black dry-erase marker and a white background behind the actuators to calculate the curvature with minimal post-processing. 

To convert pixels to centimeters, we photographed a ruler at several positions within the camera frame and concluded that the camera's distortion is tolerable in the center of the frame. If the actuator remained in the center of the camera frame, we could use a linear mapping between pixels and centimeters. 

Using OpenCV, we first converted the image to grayscale. Then, we applied a binary mask to isolate the line we drew on the actuator. Using the masked image, we found the contour line(s). Each contour is a list of $x$ and $y$ coordinates that compose the line. We add the coordinates to an array for curvature calculations for each detected contour. To calculate the average curvature, we first split the length of the contour into three sections. For each section, we used a random coordinate. For 100 iterations, we used the coordinates from each section to calculate the curvature. We averaged the 100 values to determine the mean curvature of the line drawn on the actuator. For each actuator, we measured the length of the fiberglass layer and converted curvature, $\kappa$, to bending angle using $\psi = l_{0}\kappa$. Fig. \ref{fig:opencvbendingangle} showcases the procedure for an SS40 actuator at 160~kPa; using coordinates along the contour line, we measured a bending angle of 102$^\circ$. 

\clearpage
\begin{figure}[ht]
    \centering
     \includegraphics[width=6 in]{images7/opencvbendingangle.jpg}
    \caption{Samples from the OpenCV script used to measure bending angle of the circular actuators. A. Original image of a SS40 actuator at 160kPa. B. Grayscale. C. Color mask. D. Random coordinates chosen from the contour line split into three sections and the calculated bending angle of 102$^\circ$.}
    \label{fig:opencvbendingangle}
\end{figure}

There are several drawbacks to this method of measuring bending angle. First, maintaining a white background was difficult, and white electric tape was often required to cover any markings created by the dry-erase markers. Also, we had to cover the brass barb and push-to-connect fitting with white electric tape. Additionally, we wanted to collect data from both sides of the actuator, so the dry-erase marker had to be cleaned off and redrawn on the other side after sufficient testing. The biggest drawback of this method of measuring bending angles is that the OpenCV script could not detect negative bending angles. Since the script reported the measured curvature, we had to manually indicate if the reported curvature corresponded to a positive or negative bending angle. 