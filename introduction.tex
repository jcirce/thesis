\chapter{Introduction}

Soft robotics, a relatively new field, creates robots made from soft, compliant materials. Instead of using rigid links to create degrees of freedom, soft robots use continuous, elastic deformation to achieve different types of useful motion. Materials used to construct soft robots can comply and deform to interact with their environment; soft robots are made to perform in unstructured environments \cite{lee_soft_2017}. 

Soft robots can achieve typical robot functionality of grasping and locomotion; they are also uniquely capable of squeezing, stretching, climbing, and growing \cite{laschi_soft_2016}. Soft pneumatic actuators, also known as elastic inflatable actuators, use a pressurized fluid (air or another fluid) inside an inflatable chamber. Strategically placed strain-limiting, inextensible materials around the soft body control the material's deformation \cite{zaidi_actuation_2021}. These pneumatic actuators can create motions such as expanding, contracting, twisting, or bending \cite{gorissen_elastic_2017, al-ibadi_circular_2018}. 

Soft pneumatic actuators achieve bending using strain-limiting materials to vary the longitudinal or axial strains across the cross-section upon actuation. For existing single-chamber actuators, positive pressurization induces a single-direction of bending \cite{galloway_mechanically_2013}. Similar to a finger, actuators with a single bending direction are typically straight without input pressure, and they bend and curl in with increasing pressure. Actuators with a single direction of bending have a limited range of motion and typically require more than one single-chamber actuator for grasping applications. These actuators come in many sizes; a small 1~cm long actuator was designed for minimally invasive surgical applications \cite{mccandless_soft_2022}. A bellows or pneumatic network actuator can be cast from a single pour of silicone; three of these actuators combine to form an end-effector capable of picking up delicate objects such as a pepper \cite{alici_modeling_2018}. Using flexible filament, combinations of 3-D printed bellows actuators form a soft hand \cite{sundaram_dragonclaw_2023}. These actuators, with a single direction of bending behavior, are shown in Fig. \ref{fig:singlechamber}. \\

\begin{figure}[!ht]
    \centering
    \includegraphics[width=6 in]{images1/singlechamber.jpg}
    \caption{Samples of single chamber soft pneumatic actuators. A. Fiber-reinforced soft pneumatic actuator \cite{galloway_mechanically_2013}. B. Small pneumatic actuator attached to the end of a bronchoscope \cite{mccandless_soft_2022}. C. Three bellows actuators are used to grasp a pepper \cite{alici_modeling_2018}. D. 3D-printed three-finger hand with a thumb \cite{sundaram_dragonclaw_2023}.}
    \label{fig:singlechamber}
\end{figure}

Soft pneumatic actuators with more than one direction of bending enable more types of motion from a single actuator, leading to a wider range of motion and more possibilities for applications. With access to both compressed air and a vacuum pump, a single chamber actuator can achieve bi-directional bending with positive and negative pressures \cite{wakimoto_miniature_2011, ariyanto_three-fingered_2019, fatahillah_novel_2020}. Fig. \ref{fig:posandnegpressure} contains examples of using both positive and negative pressure to achieve bi-directional bending. A miniature, 2~mm wide, 15~mm tall bellows actuator used both positive and negative pressure to achieve bi-directional bending (Fig. \ref{fig:posandnegpressure}A). Another bellows type actuator, when used in a set of three, used negative pressure to pick up a cup from the inside. (Fig. \ref{fig:posandnegpressure}B). \\

\begin{figure}[!ht]
    \centering
    \includegraphics[width=6 in]{images1/posandnegpressure.pdf}
    \caption{Soft pneumatic actuators that use positive and negative pressure to achieve bi-directional bending. A. Miniature bellows-type actuator \cite{wakimoto_miniature_2011}. B. Three actuators grasping a cup from the inside with negative pressure \cite{ariyanto_three-fingered_2019}.}
    \label{fig:posandnegpressure}
\end{figure}

Another way to achieve multi-directional bending is to increase the number, shape, and size of inflatable chambers within a single actuator \cite{bilodeau_design_2018, pagoli_soft_2021, fei_novel_2019, cappello_exploiting_2018, pang_novel_2019}. Fig. \ref{fig:multichamber} contains drawings and images of multi-chambered soft actuators. A tapered octopus arm actuator has non-constant curvature bending due to the taper and small suckers that use negative pressure to hold objects \cite{xie_octopus_2020}. A two-chamber actuator achieves constant curvature bending in two directions using two identical chambers \cite{bilodeau_design_2018}. A three-chamber actuator can bend and twist, enabling a group of three actuators to solve a Rubix cube \cite{pagoli_soft_2021}. An actuator with 20 individual chambers (four in the wrist and two in each of four fingers) enables an incredible range of motion \cite{fei_novel_2019}. These actuators can grasp small and large objects and pick up objects from the inside. \\

\begin{figure}[!ht]
    \centering
    \includegraphics[width=6 in]{images1/multichamber.jpg}
    \caption{Samples of multi-chamber soft pneumatic actuators. A. Octopus arm \cite{xie_octopus_2020}. B. Two-chambered actuator with bi-directional bending \cite{bilodeau_design_2018}. C. Three-chambered actuators solving a Rubix cube \cite{pagoli_soft_2021}. D. Side view of a 20-chambered wrist and finger actuator \cite{fei_novel_2019}.}
    \label{fig:multichamber}
\end{figure}

Although complex to fabricate and actuate, multi-chambered robots can generate many types of complex locomotion in addition to grasping objects, inspired by a fish \cite{zhou_modeling_2020}, eel \cite{feng_body_2020,nguyen_anguilliform_2022}, or snake \cite{arachchige_wheelless_2023}. These robots have sets of two-chamber pairs in series to generate different bending behaviors along the length of the actuator. Fig. \ref{fig:eelsandsnakes} contains these bio-inspired robots during actuation. These eel-inspired actuators each have eight total chambers. The snake-inspired robot has four sets of three actuators in series: 12 total chambers. \\

\begin{figure}[!ht]
    \centering
    \includegraphics[width=6 in]{images1/eelsandsnakes.pdf}
    \caption{Multi-chamber soft robots capable of locomotion. A. An-eel inspired robot \cite{feng_body_2020}. B. Another eel-inspired robot \cite{nguyen_anguilliform_2022}. C. Snake-inspired robot \cite{arachchige_wheelless_2023}.}
    \label{fig:eelsandsnakes}
\end{figure}

Increasing the number of independently actuated chambers enables a wider range of motion but requires more complex control, especially if both positive and negative pressures are required. Advancing the field requires developing soft robots with more types of achievable motion from each control input, simplifying possible applications by reducing the number of control inputs per robot \cite{gorissen_elastic_2017}. 

A recent development in developing soft pneumatic actuators with a wider range of motion without the complexity of multiple chambers or both positive and negative pressure is to change how the actuator is fabricated. Suppose we fabricate the actuator with an initial bending angle or curvature. In that case, the robot will have more efficient motion and a greater range of bending angles by reducing the material strain \cite{perez-guagnelli_deflected_2022}. 

Previous pre-curved actuators used two soft materials to create bending with positive pressure \cite{hu_precurved_2022}. With increasing pressure, these actuators could either unbend or bend further depending on the placement of the stiffer silicone, as shown in Fig. \ref{fig:precurved}. The flexing actuator has stiffer silicone on the inside so that with pressurization, the actuator increases its curvature. The counter-flexing actuator has stiffer silicone on the outside for the opposite effect. \\

\begin{figure}[!ht]
    \centering
    \includegraphics[width=6.5 in]{images1/precurved.pdf}
    \caption{Two types of a pre-curved actuator, flexing actuators close with pressure and counter-flexing actuators open with pressure \cite{hu_precurved_2022}.}
    \label{fig:precurved}
\end{figure}

This work aimed to develop and characterize a soft pneumatic actuator with bi-directional bending behavior in a simpler way. The circular soft pneumatic actuator has a single chamber made from a single silicone. We achieve bi-directional bending by fabricating the actuator in a near-circular shape: with increasing pressure, the actuator fully uncurls and then curls back on itself. A positive pressure source, the circular actuator's single control input, is all that is required for the entire range of motion; there is no need to control multiple chambers or supply positive and negative pressure. Additionally, because of its length and shape, a single actuator can grasp and pick up objects from both the inside and outside of the object. The circular actuator has the same functionality as a bi-directional robot that previously required two chambers. Since the robot is composed of a single chamber, the robot can achieve greater curvature because there is less silicone to stretch. As seen in Fig. \ref{fig:intro}, the circular soft actuator has an incredible range of motion for an actuator with one control input. \\ \\ \\

\begin{figure}[!ht]
    \centering
    \includegraphics[width=3.5 in]{images1/intro.jpg}
    \caption{A circular soft pneumatic actuator made of DragonSkin 20 silicone.}
    \label{fig:intro}
\end{figure}

This thesis details the development and characterization of the circular actuator. Chapter 2 includes initial designs and defines the circular actuator's circular shape and cross-section. Chapter 3 contains how we developed the high-volume fabrication process for the actuator and the chosen procedure and materials. Chapter 4 explains the design of the pressurization equipment used to add input pressure to the single-chambered actuator, including setups for two different microcontrollers. Chapter 5 contains the analytical model we developed to estimate the bending angle for a given input pressure. Chapter 6 details the experimental setup we developed for measuring the bending angle at each input pressure. Chapter 7 presents the results of bending angle experiments compared to the analytical model. Chapter 8 details the experimental setup we developed for measuring the blocked force of the circular actuator. Chapter 9 presents the results of blocked force experimentation and a few applications of the circular actuator. 

\clearpage
\section{Statement of Problem}

The purpose of this work is to create and characterize a single-chambered soft pneumatic actuator made from a single soft material that can generate bi-directional bending behavior using only a positive pressure source. 