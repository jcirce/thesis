\chapter{Blocked Force Results}

\section*{Overview}

This chapter contains the results of blocked force experiments with a load cell and using the actuator to grasp various objects that block the actuator's axial deformation in different ways, as described in chapter \ref{chapter:blockedforce}. Additionally, we demonstrate how the circular actuator's silicone exterior conforms to an object and can grab and hold objects without causing damage.

\section{Load Cell Results}

We measured the blocked force of the circular actuators using a method we developed, outlined in chapter \ref{chapter:blockedforce}. We measured the force created when restricting the bending angle near $\psi_0$. The range of pressures we used to measure the force was as high as the actuator could take before the developed eccentricity was high enough such that the experimental setup could no longer measure the force; the angle of the thread with respect to the load cell became large enough such that the thread would fall off the pulley. Since this maximum pressure was above the pressure expected for $\psi$ to reach $0^\circ$, we decided not to continue iterating the experimental setup. 

On average, at the maximum pressure used, the DS20 actuators achieved $2.7\pm0.2$~N of blocked force at 105~kPa. DS30 achieved $2.2\pm0.2$~N at 112~kPa. At 210~kPa, SS40 and SS50 achieved $5.1\pm0.2$ and $1.63\pm0.06$~N, respectively. While stiffer actuators could generate higher forces at higher pressures, stiffer actuators require more input pressure for a particular force. Fig \ref{fig:blockedforce} contains one standard deviation of blocked force measured for all four actuators. 

\begin{figure}[!ht]
    \centering
     \includegraphics[width=5.5 in]{images9/blockedforce.jpg}
    \caption{Blocked force measured between 0-210~kPa. The shaded region represents one standard deviation of blocked force for circular actuators of four materials.}
    \label{fig:blockedforce}
\end{figure}

Theoretically, if the material properties of the silicone are restricted such that no deformation occurs, the moment from the blocked force would be equal to the moment from pressure, as defined with Eq.~\ref{eq:moment-p}, a linear relationship \cite{polygerinos_modeling_2015}. The actuators show an almost linear relationship between blocked force and pressure for low input pressures until about 40~kPa. At this pressure, material properties increase the blocked force reading for DS20, and it breaks away from the three stiffer materials. Both DS30 and SS40 break away from the stiff SS50 at around 60~kPa. Softer silicones experiencing non-linear blocked force readings at lower pressures lead us to conclude that the non-linear force readings must be due to the material properties of the circular actuator.

Not only is there a highly non-linear relationship between stress and strain for the hyperelastic silicones, but for the circular actuators, input pressure causes strain in all three directions: axial, circumferential, and radial. The inextensible thread attached to the load cell restricts axial deformation but does not restrict circumferential or radial deformation. The circumferential expansion of the actuators from input pressure induces a proportional axial and radial deformation. However, because we are restricting axial deformation, the silicone is in compression in the axial direction, increasing the load cell reading. Increasing pressure further increases these effects, increasing the load cell readings at a higher rate. Fig. \ref{fig:compressionfbd} labels the compression in the circular actuator and the force reading on the load cell. 

\begin{figure}[!ht]
    \centering
     \includegraphics[width=2 in]{images9/compressionfbd.pdf}
    \caption{DS20 actuator during blocked force testing with the load cell, $\sigma$ is the axial compression and $F$ is the force on the load cell.}
    \label{fig:compressionfbd}
\end{figure}

When comparing the shape of the actuator at different pressures for blocked force and bending angle experiments, we can compare the blocked force reading to the unconstrained bending angle to see how pressures past $\psi=0$ that would have generated a negative $\psi$ have a higher $dF/dP$ compared to pressures for more positive bending angles.  

Fig. \ref{fig:d20force} contains photographs of a DS20 actuator at increasing input pressure (0-105~kPa) during blocked force and bending angle experiments. The orange dashed line is parallel to the inextensible thread connected between the free end of the actuator and the bearing. At 28~kPa, we measured an average of $0.2\pm0.1$~N of force and local $dF/dP=0.006$~N/kPa. At 70~kPa: $1.2\pm0.2$~N and $dF/dP=0.04$~N/kPa, and 105~kPa: $2.7\pm0.2$~N and $dF/dP=0.05$~N/kPa. At these pressures, the actuator exhibited a positive bending angle, a bending angle close to $0^\circ$, and a large negative $\psi$. 

\begin{figure}[!ht]
    \centering
     \includegraphics[width=5 in]{images9/d20forcesamples.pdf}
    \caption{Photographs of a DS20 actuator during blocked force experiments and bending angle experiments for 28, 70, and 105~kPa.}
    \label{fig:d20force}
\end{figure}

Despite not reaching $\psi=0^\circ$ until over 200~kPa during bending angle experiments, SS40 still had a non-linear relationship between pressure and blocked force. At 63~kPa, we measured SS40 $0.44\pm0.6$~N and $dF/dP=0.01$~N/kPa. At 139 kPa, $1.95\pm0.07$~N and $dF/dP=0.04$~N/kPa, and at 210 kPa $5.1\pm0.2$~N and $dF/dP=0.05$~N/kPa. 

\clearpage
During blocked force testing, we photographed the circular actuators to measure the eccentricity of the elliptical arc formed at the maximum pressure. Fig. \ref{fig:eccentricity} contains photographs of circular actuators of each material at 0~kPa and the maximum pressure used during blocked force testing. The yellow dashed line represents the ellipse we used to calculate the eccentricity of the actuator. At the maximum pressure of 105~kPa, the DS20 actuator displayed an eccentricity of 0.6, while at the higher pressure of 120~kPa, the DS30 actuator displayed a lower eccentricity of 0.5. At an even higher pressure of 210~kPa, the SS40 and SS50 actuators displayed 0.55 and 0.4, respectively. 
\\
\begin{figure}[!ht]
    \centering
     \includegraphics[width=6.5 in]{images9/eccentricity.jpg}
    \caption{Photographs of circular actuators during blocked force testing at 0~kPa and the maximum pressure for each material. The yellow dashed line represents the ellipse used to measure eccentricity, $e$.}
    \label{fig:eccentricity}
\end{figure}

\section{Picking up a Cup}

Using an object with an internal diameter of 9~cm, smaller than the $\psi_0$ bending radius, we placed circular actuators inside the object. With increasing pressure, we measured the maximum weight the actuators could hold for an object of this size. This experiment explores how adding internal pressure to the actuator increases the blocked force generated when constrained at a certain eccentricity ( $\approx0.7$). For each material, at each pressure increment, we added more weight (using 50~g increments) inside the object until the actuator slipped out of the object when picked up from the center of the actuator. Fig. \ref{fig:coopercupphotos} contains samples of actuators of each material holding increasing weight at different pressures inside a cup. For objects with an inner diameter requiring us to compress the actuator so that it can make contact with both walls of the object, the compression of the actuator itself generates a blocking force on the wall of the object. For the stiffness SS50 actuators, at just 20~kPa, the actuator could pick up 1000~g of weight inside the cup, a significant increase compared to the other materials. 
\\
\begin{figure}[!ht]
    \centering
     \includegraphics[width=6.5 in]{images10/coopercupphotos.jpg}
    \caption{Photographs of circular actuators with internal pressure to pick up a cup with increasing weight.}
    \label{fig:coopercupphotos}
\end{figure}

DS20, DS30, and SS40 actuators held 700~g within a pressure range of 0-60~kPa when placed inside the cup. Increasing material stiffness could hold about 50~g more weight for a given input pressure. Fig. \ref{fig:coopercupdata} contains the weight actuators could hold inside the cup for pressures between 20-90~kPa. At pressures higher than 60~kPa, the DS20 actuator buckled in the center and could not maintain contact with the inner walls of the object. The DS30 and SS40 actuators at 80~kPa held 750~g and 800~g, respectively. \\ 

\begin{figure}[!ht]
    \centering
     \includegraphics[width=5 in]{images10/coopercupdata.jpg}
    \caption{Maximum weight inside an object of 9~cm inner diameter actuators of three materials could hold without slipping.}
    \label{fig:coopercupdata}
\end{figure}

\clearpage
\section{Round Objects}

For a volleyball, as shown in Fig. \ref{fig:aroundvolleyball}, a DS20 circular actuator, when held against the volleyball, can conform to the size of the ball and, with enough pressure, generate enough force on each end to be able to pick up the ball when held in the center. The second photograph is mid-pressurization before the volleyball restricts the actuator's bending behavior.

\begin{figure}[!ht]
    \centering
     \includegraphics[width=6 in]{images10/aroundvolleyball.jpg}
    \caption{Photographs of a DS20 circular actuator with increasing pressure picking up a volleyball.}
    \label{fig:aroundvolleyball}
\end{figure}

We used a DS30 actuator and a round container with a 15~cm diameter to show one of the unique properties of circular actuators: they can grasp objects from both the inside and the outside. When placed inside the container, we pressurized the DS30 actuator to 110~kPa, which generated a low enough $\psi$ so that the container walls blocked the actuator. We added weights (clamps and fasteners) to the inside of the container to showcase the actuator's strength. At 110~kPa, the DS30 actuator could hold 380~g inside the container. To hold the container from the outside, we used an input pressure of 165~kPa, the minimum required pressure to grasp this sized object with a negative $\psi$. We found that the DS30 actuator could hold 420~g inside the container at this pressure. The DS30 actuator also could pick up a volleyball (270~g, outer diameter of 20~cm) at 150 kPa. Fig. \ref{fig:ds30roundobjects} contains photographs of a DS30 actuator holding the container and the volleyball. We used Kevlar thread placed around the center of the actuator to pick up the objects so as not to add external forces to the ends of the actuator.
\\
\begin{figure}[!ht]
    \centering
     \includegraphics[width=6.5 in]{images10/ds30roundobjects.jpg}
    \caption{Photographs of a DS30 circular actuator picking up a container with 15~cm diameter and a volleyball of 20~cm diameter. The input pressures and weight of the objects are labeled.}
    \label{fig:ds30roundobjects}
\end{figure}

\clearpage
\section{Interacting with Objects}

The circular actuator can interact with objects of any shape without causing damage to the object, demonstrating its viability as a soft robot. For a hollow 3D-printed dodecahedron, the circular actuator, once placed inside, conforms to the object's walls when pressurized. With increasing pressure, the actuator's curvature and circumferential expansion conform to the object, and with pressures inducing a negative $\psi$, the actuator can hold the object above itself. Fig. \ref{fig:3dprintedonend} contains photographs of a DS20 circular actuator with increasing pressure conforming to and picking up a 3D-printed hollow dodecahedron. 

\begin{figure}[!ht]
    \centering
    \includegraphics[width=6.5 in]{images10/3dprintedonend.jpg}
    \caption{Photographs of a DS20 circular actuator with increasing pressure picking up a hollow 3D-printed dodecahedron.}
    \label{fig:3dprintedonend}
\end{figure}

For objects smaller than the initial bending radius of the circular actuator, simply placing the unpressurized actuator on one side of the object is enough to grab the object. For a box of gloves, a DS20 actuator can self-align with the object with increasing pressure. Fig. \ref{fig:boxofgloves} contains photographs of a DS20 actuator picking up a box of gloves. For this object, we stepped from 0 kPa to a pressure high enough to induce a negative $\psi$ based on the size of the box. The first photograph is of the actuator resting on the box. The second and third photographs are of the actuator mid-pressurization flipping its orientation. The fourth is once the actuator reaches the desired input pressure. Holding the actuator from the center can easily pick up the box. Generating sufficient $\psi$ to conform around the object from rest with increasing pressure without human input showcases the versatility of the circular actuator. 

\begin{figure}[!ht]
    \centering
     \includegraphics[width=6.5 in]{images10/boxofgloves.jpg}
    \caption{Photographs of a DS20 circular actuator with increasing pressure picking up a box of gloves.}
    \label{fig:boxofgloves}
\end{figure}
