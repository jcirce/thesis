\chapter{Analytical Model For Bending Angle}
\label{chapter:model}

% \section{Analytical Model for Bending Angle}

To characterize the actuator's bending behavior as a function of input pressure and material properties, we developed an analytical model for curling and uncurling based on those found in \cite{polygerinos_modeling_2015, connolly_automatic_2017, hu_precurved_2022}. We calculate the input pressure $P$ required to reach a given bending angle $\psi$ using the geometry of the actuator, the length of the inextensible layer $l_0$, and the non-linear hyperelastic material properties of each silicone, including non-axial deformation. To account for circumferential deformation, we transform the cross-section at each pressure based on material stretches. As the pressure increases, the actuator uncurls (decreasing $\psi$) from its initial bending angle $\psi_0$ to an angle $\psi=0$, where the actuator is straight. Further increasing pressure causes $\psi$ to become negative, corresponding to curling behavior, thus generating a bi-directional range of motion. Fig. \ref{fig:modelall} contains all of the variables used in the analytical model. 

\begin{figure}[ht]
    \centering
     \includegraphics[width=6 in]{images5/modelall.jpg}
    \caption{Geometric variables defined for the analytical model. The thicker line represents the inextensible layer. A. Definition of positive and negative bending angles. B. Undeformed cross-sectional geometry of the actuator. C. Axial stretch ratios and corresponding stress in response to applied pressure. D. Deformed cross-sectional geometry to account for circumferential and radial deformation.}
    \label{fig:modelall}
\end{figure}

\clearpage
\section{Stretch Ratios}

As the actuator bends, the silicone stretches in the axial direction to maintain the length $l_0$ of the inextensible layer. The model assumes that the actuator maintains a single, constant bending radius along its length at all pressures and that the cross-section remains orthogonal to the inextensible layer. We also assume incompressibility in the hyperelastic material, such that $\lambda_{1}\lambda_{2}\lambda_{3} = 1$ where $\lambda_i$ is the stretch ratio in each direction (axial, circumferential, and radial, respectively). This assumption requires that the cross-sectional area of the silicone remains constant, even as the shape deforms. Increasing pressure causes the semi-circular region to expand outwards into a circular shape, resulting in thinning of the actuator's walls to maintain a constant cross-sectional area (Fig. \ref{fig:modelall}D). We assume no geometric changes in the base layer because of the proximity to the inextensible layer.

For the undeformed cross section, as shown in Fig.~\ref{fig:modelall}B, $\beta$ is the coordinate in the base layer away from the inextensible layer, and $\tau$ is the distance from the inner wall in the $\varphi$ direction. The base region has thickness $b$~=~0.38~cm, and the undeformed semi-circular cross-section wall has radius $a_0$~=~0.64~cm and thickness $t_0$~=~0.38~cm. In the deformed cross-section (Fig.~\ref{fig:modelall}D), we assume that the silicone forms a circular arc with offset from the base $q$, inner radius $a$, wall thickness $t$ and angle range $-\varphi_m$ to $\varphi_m$. $\tau'$ and $\varphi'$ are scaled coordinates with reference to the undeformed shape such that $\tau'=\tau\frac{t}{t_0}$ and $\varphi'=\varphi\frac{2\varphi_m}{\pi}$.

For a given bending angle $\psi$, we calculate the axial stretch ratio $\lambda_1$ as $\lambda_{\beta}$ for material in the base and $\lambda_{\varphi,\tau}$ in the semi-circular region (Fig.~\ref{fig:modelall}C). When $\psi$ is negative, the axial stretch ratio remains positive as all lengths must exceed $l_0$. We also calculate the circumferential stretch ratio based on the deformed cross-section. These equations hold for the entire range of $\psi$ and deformation of the cross-section. 

\begin{equation}
    \lambda_1=\lambda_{\beta} = \frac{l_{0} - \beta \psi}{l_{0}-\beta\psi_0} 
    \label{eq:stretchbeta}
\end{equation}

\begin{equation}
    \lambda_1=\lambda_{\varphi,\tau} = \frac{l_{0} - (b+q+(a+\tau')\cos\varphi')\psi}{l_{0} - (b+(a_0+\tau)\cos\varphi)\psi_{0}}
    \label{eq:stretchphitau}
\end{equation}

\begin{equation}
    \lambda_2=\lambda_c = \frac{2(a+\tau')\varphi_m}{(a_0+\tau)\pi} 
    \label{eq:stretchc}
\end{equation}

For the undeformed shape, the circumferential stretch ratio $\lambda_2$ is 1. The stretch ratio in the radial direction $\lambda_3=\lambda_r$ can be calculated as $\lambda_3=1/(\lambda_1 \lambda_2)$ from the incompressibility assumption. To find the wall thickness $t$ of the circular arc region of the deformed cross-section, we integrate the radial stretch of the undeformed shape over the thickness:

\begin{equation}
    t = \int_0^{t_0} \frac{1}{\lambda_{\varphi,\tau} \times 1} d\tau 
    \label{eq:thicknessfromlambdar}
\end{equation}

To account for the inaccuracy of the material models and the compressibility of the DragonSkin materials, we scale the calculated thickness by $k_t(1-|\psi-\psi_0|/\psi_0)$ where $k_t = 0.3$ is a constant determined by best fit to the experimental expansion results. This scaling accounts for the non-zero radial stress, especially at negative bending angles. No scaling is necessary for the stiffer SmoothSil materials. 

To transform the cross-section from the original semi-circle to the circular arc, we use the scaled thickness $t$ and the cross-sectional area $A_0$ to calculate the deformed inner radius $a$, angle range $2\varphi_m$, and vertical offset $q$:

\begin{equation}
    \frac{A_0}{2\sin{\varphi_m}t}-\frac{t}{2}=\frac{a_0}{\sin{\varphi_m}}
    \label{eq:phi_m}
\end{equation}

\begin{equation}
    a=\frac{A_0}{2t\varphi_m}-\frac{t}{2}
    \label{eq:a_aug}
\end{equation}

\begin{equation}
    q=-a\cos{\varphi_m}
    \label{eq:q_aug}
\end{equation}

The axial and circumferential stretch ratios are then recalculated using the deformed cross section and Equations \ref{eq:stretchbeta}--\ref{eq:stretchc} before calculating stress with the material models.

\section{Converting Strain to Stress}

To describe the stress-strain relationship for each of the soft materials, multiple material models are available; they can vary significantly in accuracy depending on the amount of strain \cite{paterno_hybrid_2018}. We choose the Ogden hyperelastic material model because it is more accurate at higher stretches and for softer materials \cite{marechal_toward_2021}. The strain energy density function $W$ for the Ogden model is given by Eq. \ref{eq:strain-ogden}, where $\mu_{p}$ and $\alpha_{p}$ are experimentally determined material coefficients. We can use a given Neo-Hookean material model coefficient ($\mu$) in the Ogden model by setting $N=1$ and $\alpha_1=2$. 

\begin{equation}
    W=\sum_{p=1}^N \frac{\mu_p}{\alpha_p}(\lambda_1^{\alpha_p}+\lambda_2^{\alpha_p}+\lambda_3^{\alpha_p}-3)
    \label{eq:strain-ogden}
\end{equation}
\\
The Cauchy stresses in all three directions are given by $\sigma_j=-p+\lambda_j\frac{\partial W}{\partial\lambda_j}$, where $p$ is a Lagrange multiplier. The stresses in the radial direction are significantly smaller than those in the other directions, so we assume $\sigma_3=0$. We calculate the axial and circumferential stresses using Equations \ref{eq:s_1} - \ref{eq:multiplier-p} and the material properties found in Table \ref{table}. 

\begin{equation}
    \sigma_1 =-p+\mu_1\lambda_1^{\alpha_1}+\mu_2\lambda_1^{\alpha_2}+\mu_3\lambda_1^{\alpha_3}
    \label{eq:s_1}
\end{equation}

\begin{equation}
    \sigma_2 =-p+\mu_1\lambda_2^{\alpha_1}+\mu_2\lambda_2^{\alpha_2}+\mu_3\lambda_2^{\alpha_3}
    \label{eq:s_2}
\end{equation}

\begin{equation}
    p=\mu_1(\lambda_1\lambda_2)^{-\alpha_1}+\mu_2(\lambda_1\lambda_2)^{-\alpha_2}+\mu_3(\lambda_1\lambda_2)^{-\alpha_3}
    \label{eq:multiplier-p}
\end{equation}

\begin{table}[h!]
\caption{Material Properties Used in Analytical Model}
\label{table}
\centering
\begin{tabular}{l r r l}
\hline
{} & \bfseries Variable & \bfseries Value & \bfseries Unit \\
\hline\hline
DS20 \cite{marechal_toward_2021} & $\mu_1$, $\mu_2$, $\mu_3$ & -0.9534, -1.4515, 2.4085 & MPa\\
{} & $\alpha_1$, $\alpha_2$, $\alpha_3$ & 3.478, 3.181, 3.339 & {}\\
\hline
DS30 \cite{marechal_toward_2021} & $\mu_1$, $\mu_2$, $\mu_3$ & 0.03816, 0.02524, 0.04456 & MPa\\
{} & $\alpha_p$ & 3.417 & {}\\
\hline
SS40 \cite{pagoli_review_2022} & $\mu_1$ & 0.24 & MPa\\
{} & $\alpha_1$ & 2 & {}\\
\hline
SS50 \cite{xavier_finite_2021} & $\mu_1$ & 0.68 & MPa\\
{} & $\alpha_1$ & 2 & {}\\
\end{tabular}
\label{table}
\end{table}

\clearpage
\section{Pressure from Bending Moments}

The bending moments $M_P$ and $M_\sigma$ about the \(O\) axis (Fig.~\ref{fig:modelall}C) result from pressure on the end of the actuator and axial stress, respectively. These moments must be equivalent for the actuator to achieve static equilibrium. We calculate the bending moment $M_\sigma$ as the integral of the force from axial stress in both the base ($\sigma_{\beta}$) and the circular arc ($\sigma_{\varphi, \tau}$) regions times the orthogonal distance from \(O\) for the entire length of the actuator. 

\begin{equation}
    M_{\sigma} = 2l_{0} \int_{0}^{b} \sigma_{\beta} (a_0+t_0) \beta  d\beta
    + l_{0}\int_{0}^{t_0}\hspace{-2px}\int_{-\frac{\pi}{2}}^{\frac{\pi}{2}}\hspace{-2px}\sigma_{\varphi,\tau}((a+\tau')\cos{\varphi'}+b+q)(a+\tau') d\varphi d\tau
    \label{eq:moment-axial}
\end{equation}

The bending moment $M_P$ resulting from a given input pressure \(P\) acting on the end of the actuator is a function of the deformed cross-sectional geometry: 

\begin{align}
    M_P & = P \int_{-\frac{\pi}{2}}^{\frac{\pi}{2}}\int_0^{a} a(b+q+a\cos{\varphi})d\varphi da \\ \nonumber \\
    & = \frac{4 a^{3} + 3 \pi a^{2} (b+q)}{6} P 
    \label{eq:moment-p}
\end{align}

We equate (\ref{eq:moment-axial}) and (\ref{eq:moment-p}) and solve for the required input pressure $P$ to induce the axial stresses for a given bending angle $\psi$. To account for the underestimation of $P$ due to uncertainty in the material models and non-uniformity in the actuator, we scale the calculated pressure by a constant $k_P$, determined by a best fit for each material. $k_P$ is 3.8 for DS20, 1.8 for DS30, and 2.2 for SS40 and SS50. 