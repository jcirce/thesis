\chapter{Conclusions and Recommendations}

This work has presented a circular soft pneumatic actuator with bi-directional bending behavior. We fabricated the single-chamber actuator out of a uniform soft material. We successfully fabricated several circular actuators using the simple and high-volume fabrication process we designed. We constructed the pressurization equipment with an air tank that automatically filled itself and designed electrical hardware to control the input pressure to the actuators. We developed an analytical model for bending angle that includes non-axial deformation and was accurate to experimental data. We developed the testing protocols and automated calculations of the bending angle for each photograph taken and presented the results, showing the bi-directional bending behavior. We developed an experimental setup for measuring the blocked force when restricting the circular actuator's bending behavior. We also presented grasping results of a single circular actuator holding objects from the inside and outside. We did all of this with silicones of four different shore A hardness, further emphasizing the generalizability of the fabrication process, characterization methods, and validity of the circular actuator's design. \\

Further exploration of the forces generated from blocking circular actuators at high eccentricity and modeling the elliptical arc formed when blocking the actuator at different points along its length will allow for a complete model of the circular actuator's blocked force behavior. Additionally, exploring varying the cross-section area over length would allow for non-constant curvature bending behavior, making the actuator more applicable for bio-inspired applications. 

Compared to existing soft pneumatic actuators, the circular actuator has a more straightforward fabrication process and has a greater range of bending behavior. Where previous applications require multiple actuators to grasp objects from the inside, a single circular actuator accomplishes the same goal. Fabricating soft pneumatic actuators with initial bending angles or curvature allows for a simpler solution for grasping and locomotion than actuators that previously required several individual chambers. 

Scaling down the circular soft actuator would not influence its capabilities for bidirectional bending behavior. The same fabrication process can be used, and smaller objects would be within the actuator's capabilities. The circular actuator would be an excellent end effector when held on one end in applications requiring vast bending behavior or a stiff actuator with a low bending angle. If held on both ends, we could control the eccentricity and bending radius. A single circular actuator could replace multiple actuators when grasping an object from the inside. High pressurization is not required for circular actuators made from stiffer silicones to develop a significant grasping force when placed inside objects.

When designing new soft robots, it is essential to question overly complex fabrication methods and the applicability of actuators requiring dozens of chambers. The future of soft robotics lies in finding simpler solutions to achieve the same behavior, like the circular actuator: a bi-directional bending actuator made from a single material with a single chamber.  