\chapter{Introduction}
\section{Statement of Problem}
Soft robots are safer for human interaction than traditional robots and are often used in grasping applications due to their inherent compliance and adaptability \cite{laschi_soft_2016, hotoda_octopus-like_2023}. Soft pneumatic actuators are a subset of soft robots that generate motions such as bending, twisting, and extending with changes in internal pressure. These behaviors rely on pre-programmed asymmetries in material stiffness or geometry \cite{galloway_mechanically_2013, mccandless_soft_2022, ellis_generative_2022}. Simple bending pneumatic actuators generally consist of an inflatable chamber constrained in one or more directions by inextensible materials, resulting in motion along a single degree of freedom. Multiple simple chambers can be connected in parallel and actuated independently to achieve multi-directional bending \cite{bilodeau_design_2018, pagoli_soft_2021}. Placing several of these multi-chambered actuators in series allows for an extensive range of motion with multiple degrees of freedom \cite{feng_body_2020, nguyen_anguilliform_2022, arachchige_wheelless_2023}. Many soft pneumatic actuators are capable of grasping objects in multiple orientations by using the additional degrees of freedom from multiple chambers \cite{fei_novel_2019} or combinations of positive and negative pressure \cite{ariyanto_three-fingered_2019}. 

Increasing the number of independently actuated chambers enables a wider range of motion but requires more complex control. Increasing the bending capabilities with fewer chambers can require two regulators to generate both positive and negative pressure \cite{fatahillah_novel_2020}. Advancing the field requires developing soft robots with more types of achievable motion from each control input, simplifying possible applications by reducing the number of control inputs per robot \cite{gorissen_elastic_2017}.

Initially-straight actuators are generally easy to fabricate but are limited in their useful range of motion. Fabricating actuators with curvature or an initial bending angle allows for more efficient motion and a greater range of bending angles by reducing the material strain \cite{perez-guagnelli_deflected_2022}. Previous examples of circular pneumatic actuators had a closed circle configuration and could grasp objects but could not bend \cite{al-ibadi_circular_2018}. Other pre-curved actuators consisting of a single chamber composed of two silicones could only uncurl or curl further based on which side has stiffer silicone \cite{hu_precurved_2022}. These actuators generate either curling or uncurling motion with a single control input; however, they require advanced fabrication techniques. No single-chamber soft pneumatic actuator made of a uniform soft material has demonstrated both uncurling and curling bending behavior only with positive pressure input. 

% We present a circular soft pneumatic actuator with bi-directional bending from a single positive pressure source. As shown in Fig. \ref{figure:photo-range}, with increasing pressure, the circular soft pneumatic actuator fully uncurls and then curls back on itself. Our actuator achieves this large workspace by pressurizing just one chamber using a single control signal. This allows for grasping both by uncurling inside the object and by curling around the outside of the object if the actuator is supported from its center.

% In this paper, we describe the fabrication process and present an analytical model for all bending angles as a function of input pressure, including circumferential and radial deformation. Additionally, we present an experimental procedure for characterizing the bending behavior and blocked force of actuators made from four materials. We also demonstrate force-based grasping from both the inside and outside of objects. 