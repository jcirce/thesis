\chapter{Measuring Bending Angle}

\section{Iterations on Orientation}

As with everything related to this project of designing and characterizing a new type of soft robotic actuator, discovering how to repeatably measure the bending angle at different pressures took several iterations. Even with the just the preliminary models, we knew that the actuator would unbend with increasing pressure, we needed to develop a way to hold the actuator in front of the camera so we could measure the bending angle and compare to the model. 

The curvature and calculated bending angle are measured from the fiberglass fabric embedded into the actuator, so we knew that the fabric must be perpendicular to the camera's view. We also wanted to capture the actuator's circumferential expansion in the photographs. Before we developed the brass barbs, which provided a part of the actuator we could easily clamp to, we needed to figure out a way to fix the actuator in front of the camera so that we could capture the bending behavior. 

\subsection{First Testing Orientation}

For the very first test, the materials available were a scrap piece of matte black acrylic, a triangle that was fastened to the lab table, and a piece of aluminum extrusion used to hold the acrylic upright against the triangle. The 10-32 female standoff used to inflate the actuator fit inside of the hole of the triangle and we ran the air supply tube through the triangle to hold the actuator in place. The camera was placed far enough away from the actuator so that the pressure reading (in psi) on the regulator was visible in the photo because we were still working on the communications between the regulator and the microcontroller. The first actuator photographed was DS20 silicone fabricated with a two-part insert split in the center of the actuator. To capture the bending behavior, we used $1.0~\pm~0.1$ increments from 2.0 to 12.0 psi. The photographs taken are in Fig. \ref{fig:firstdataset}. 

We identified several problems with capturing the bending angle using this orientation. Based on the model, we expected the actuator to have constant curvature at each pressure, and having the fiberglass side rest on the flat table introduced reductions in curvature, the portion of the silicone touching the table was flatter than the rest of the actuator. Additionally, the end of the actuator was pulled down by gravity, increasing the curvature, inducing a larger bending angle, $\varphi$. 

\begin{figure}[h]
    \centering
     \includegraphics[width=6 in]{images7/firstdataset.jpg}
    \caption{First photos taken to observe the bending angle of the circular actuator. Pressure increments are $1.0~\pm~0.1$ from 2.0 to 12.0 psi.}
    \label{fig:firstdataset}
\end{figure}

\clearpage

The effects of gravity in this testing orientation were worse on the stiffer silicones. Fig. \ref{fig:ss45fallsover} contains 3 sample images at 2.0, 9.2, and 12.8 psi of an SS45 actuator photographed in the first testing configuration. Note the actuator fell over while taking the middle photograph. Even though we did not fully model the SS45 silicone actuators, mostly due to lack of material properties for this material, we were able to capture the difference between 12 psi for a DS20 actuator compared to one with over double the shore hardness. 

\begin{figure}[h]
    \centering
     \includegraphics[width=6 in]{images7/ss45fallsover.jpg}
    \caption{Photos of a SS45 actuator at 2.0, 9.2, and 12.8 psi.}
    \label{fig:ss45fallsover}
\end{figure}

Despite these photographs being unusable for capturing the bending behavior, we did learn a few things from how the circular actuator behaves. The discoloration near the center of the actuator is from the Teflon tape used to seal the gap between the two halves of the insert. What we did not know at the time was that the silicone was not curing all the way in the areas the tape touched the silicone. Additionally the minor change in wall thickness in this region of silicone would cause significant variations in circumferential and radial expansion in the center of the actuator. This variation led to more circumferential expansion, causing more axial strain, which caused an increase in curvature in the center of the actuator, further reducing the constant curvature expected from the model. Moving the split of the insert to the edge of the actuator, an area of silicone that would be cut away before sealing the ends, significantly improved the bending behavior. 

\subsection{First Horizontal Testing}

In an attempt to remove the effects of gravity on the curvature of the actuator, we placed the camera pointing down at the table. This testing orientation did remove the inconsistencies created from having the actuator rest flat on the table and gravity pulling the end, but now friction was significantly impacting the bending behavior. The bending angle varied significantly compared to testing vertically. Fig. \ref{fig:ds20h} contains images of a DS20 actuator with increasing psi. 

\begin{figure}[h]
    \centering
     \includegraphics[width=6 in]{images7/ds20h.jpg}
    \caption{Photos taken of a DS20 actuator at various psi pressures when placed horizontally on the lab bench.}
    \label{fig:ds20h}
\end{figure}

\subsection{Unbending with Gravity}

In an attempt to remove friction from restricting the decrease in bending angle with increasing pressure, we entertained adding the effects of gravity to the model as an additional bending moment effecting the curvature of the actuator. After testing in this configuration, we quickly determined that gravity induced non-trivial variations in the curvature along the length of the actuator. Especially at lower pressures, the actuator lost its circular shape due to the effects of gravity. At higher pressures, the actuator was able to hold the straight position. During these tests, we were still not going above 14.6 psi when $\varphi=0$ because we had not yet discovered the negative bending angle capabilities of the circular actuator. Fig \ref{fig:withgravity} contains photographs of a DS20 actuator held at the top so it could unbend with the help of gravity. This test had removed the effects of friction because nothing was touching the actuator except the clamp on the brass barb. 

Comparing DS20 and DS30, at the same pressure values, DS20 has a lower bending angle, which is to be expected because the silicone requires more stress to achieve the same strain. Fig \ref{fig:d30withgravity} contains photographs of a DS30 actuator at the same pressure increments. This DS30 actuator displayed some twisting motion as the pressure increased, this is likely due to the fibers of the fiberglass fabric not being aligned with the actuator or the insert could have been slightly misaligned in the mold, causing uneven wall thickness which would allow for the twisting behavior. 

\begin{figure}[h]
    \centering
     \includegraphics[width=6 in]{images7/withgravity.jpg}
    \caption{Photos taken of a DS20 actuator at various psi pressures when held vertically.}
    \label{fig:withgravity}
\end{figure}

\begin{figure}[h]
    \centering
     \includegraphics[width=6 in]{images7/d30withgravity.jpg}
    \caption{Photos taken of a DS30 actuator at various psi pressures when held vertically.}
    \label{fig:ds30withgravity}
\end{figure}

\clearpage
\subsection{Vertical but Above Table}

After testing with gravity assisting the bending behavior, we attempted once again to photograph the bending angle vertically. This time, we raised the actuator using standoffs, to reduce the restrictions on curvature from previous experiments in this orientation. Additionally, the test script was modified to both increase and decrease the pressure inside the actuator so that we could capture the hysteresis of the bending angle. Unfortunately, now that the table was not there to support the actuator, gravity not only pulled the actuator so that it touched the table at low $\varphi$, but added an incredible hysteresis to the bending behavior. Fig \ref{fig:ds20sver} contains photographs of a DS20 actuator against a blue and black checkerboard background. The images on the left are increasing the pressure and the images on the right are decreasing the pressure. While decreasing the pressure, the actuator maintained significantly lower $\varphi$ than previously observed for when increasing the pressure. 

\begin{figure}[h]
    \centering
     \includegraphics[width=6.5 in]{images7/ds20sver.jpg}
    \caption{Photos taken of a DS20 actuator vertically but offset from the table with standoffs. The left group was increasing the pressure, and the right group was from decreasing the pressure.}
    \label{fig:ds20sver}
\end{figure}

\clearpage
\subsection{Unbending Against Gravity}

In yet another attempt at finding an orientation where we could measure the bending angle of the circular actuator, without the effects of friction or restrictions from the table, we tested actuators unbending against gravity. During testing in this configuration, we accidentally put 25 psi into a DS20 actuator. We were in fear the actuator would explode, but it did not! The actuator simply continued bending past $\varphi=0^\circ$ and continued what we defined as the negative $\varphi$ bending range. As novel as this discovery was, this testing configuration still induced incredible hysteresis from the effects of gravity, so we could not compare the now positive and negative bending angles to the model. 

From here on in this work, kPa will be the unit of pressure. (1.0 psi = 6.89 kPa). For DS20 actuators tested in this orientation, we took more photos around 70-90 kPa, as the actuator's bending was unstable, similar behavior to an inverted pendulum. We still found a large hysteresis in this orientation, and the unstable-inverted-pendulum-against-gravity behavior would not align with the model we developed for the bending angle. Additionally the non-constant curvature near the fixed end of the actuator made this testing orientation not ideal. Fig. \ref{fig:d20againstgravity} contains photographs of a DS20 actuator displaying its full range of bending motion against gravity. The instability and large hysteresis between 70-90 kPa led us to not continue with this testing orientation. 

\begin{figure}[h]
    \centering
     \includegraphics[width=6.5 in]{images7/d20againstgravity.jpg}
    \caption{Photographs of a DS20 actuator across its entire bending range of motion. The left group was increasing pressure, and the right group was from decreasing pressure. NOTE: Pressure values are in kPa.}
    \label{fig:d20againstgravity}
\end{figure}

\clearpage
\section{Orientation for Photographing Bending Angle}

