\chapter{Design}
\section{\emph{Reverse} Actuation}

When we started this project, the idea was to create a soft robot with \emph{reverse} the typical actuation. If the standard soft robot bends with actuation, this one would unbend or uncurl. We were inspired by butterfly proboscis, the robot would initially occupy a small area and with inflation could extend outwards. This actuation is the opposite of a standard soft pneumatic actuator which are fabricated straight and bend or curl in with actuation. Similar to a inflatable party horn, we wanted the soft robot to extend so that it could build a structure or hold something together. 

Starting with this lofty goal, we looked into how to define bending behavior for a soft robot. If the robot is to start coiled onto itself and lose curvature with actuation, we needed a way to measure said curvature. Thanks to the guidance of Professor Smyth, we created a sketch for an actuator with linearly increasing curvature along its length, where the factor controlling the slope of curvature would change with actuation. Figure \ref{fig:smythmodel} contains samples from the sketch of an actuator with varying curvature. 

\begin{figure}
    \centering
    \includegraphics[width=4 in]{images3/smythmodel.jpg}
    \caption{A sketch of decreasing curvature for a line of linearly changing curvature}
    \label{fig:smythmodel}
\end{figure}


Initially, we fabricated a few robots based on Harvard's Soft Robotics Toolkit. We found the fabrication process, requiring multiple pours of silicone to be rather tedious. 

\section{Circular Design}
% chosen size, dimensions, cross section