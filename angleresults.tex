\chapter{Bending Angle Results}
\label{chapter:angleresults}

\section*{Overview}
This chapter presents the results of bending angle experiments defined in chapter~\ref{chapter:bendingangle} compared to the analytical model defined in chapter~\ref{chapter:model}. We tested actuators made from four soft materials using the fabrication method defined in chapter~\ref{chapter:fabrication}.

We measured the bending angle of circular actuators made from four soft materials with both increasing and decreasing pressure using $7\pm1$~kPa increments (about 1 psi). We ran six tests for each actuator and took the average value of the bending angle for all of the actuators for each material (two DS20, three DS30, three SS40, and one SS50). Fig. \ref{fig:allmaterialvsmodel} contains bending angle results for actuators of all four materials. We present results for each material in a different color, and the shaded region represents one standard deviation of bending angle for increasing and decreasing pressure, using darker and lighter shading, respectively. Additionally, we added the bending angle predicted by the analytical model for the bending angle for each material.

\begin{figure}[ht]
    \centering
     \includegraphics[width=6.5 in]{images9/allmaterialvsmodel.jpg}
    \caption{Measured bending angle for circular actuators fabricated from four materials compared to analytical model. The shaded regions represent one standard deviation of bending angle for both increasing and decreasing pressure increments.}
    \label{fig:allmaterialvsmodel}
\end{figure}

\section{Overall Trends}

With increasing pressure, the bending angle evolved from the initially positive $\psi_0$, past $\psi=0$ when the actuator was straight, to negative bending angles, $\psi<0$, as the actuator curled back on itself. At angles near $\psi_0$, we observed little change with applied pressure as the axial and circumferential strains remained low. As the actuator approached and passed the straight $\psi=0$ point, demonstrated in the softer DS20 and DS30 actuators, rapid changes in bending angle occurred with small changes in pressure. 

Both the DS20 and DS30 actuators exhibit a total range of motion of more than double the initial bending angle within the given pressure range: 436$^\circ$ in 112~kPa and 424$^\circ$ in 175~kPa, respectively. The SS40 and SS50 actuators achieve 265$^\circ$ and 64$^\circ$ respectively in 255~kPa, the upper limit for the experimental setup.

As the bending angle decreased below $\psi=0$, the actuator's effective inner radius, $a$, grew rapidly. The circumferential expansion was no longer uniform along the length of the actuator, expanding more in the center than at the ends. This deformation created a non-constant curvature. Therefore, the larger uncertainty and asymptotic behavior at pressures above 80~kPa for DS20 and 140~kPa for DS30 could be attributed to enforcing the constant curvature assumption when calculating the predicted bending angle. 

The analytical model correctly captured the bi-directional bending behavior of the actuators, particularly the change in concavity between lower and higher strains. The RMS error between the predicted bending angle and the average experimental data, including both increasing and decreasing pressure increments, is 70$^\circ$ for DS20, 70$^\circ$ for DS30, 30$^\circ$ for SS40, and 5$^\circ$ for SS50. The larger error in the softer materials was a function of both the large hysteresis in the experimental results and the inherent uncertainty in the material models. While increasing the pressure inside the actuator, reducing the bending angle requires more pressure. After taking the actuator to the peak pressure, decreasing pressure increments meant that the actuator required less pressure for a given bending angle due to the hyperelastic material properties. Additionally, assuming incompressibility yields an underestimation of the required pressure for positive bending angles and an overestimation for negative bending angles. 

\clearpage  
\section{DS20}

DS20 actuators, the softest silicones we used in this work, had a total range of 436$^\circ$ in 112~kPa. On average, starting at an initial bending angle, $\psi_0=210^\circ$, with increasing pressure, the DS20 actuators reached $\psi=0^\circ$. They continued bending with a negative bending angle, reaching past $\psi<-\psi_0$ to a bending angle of $-226^\circ$. DS20 actuators had significant hysteresis (up to 160$^\circ$) in bending angle between increasing and decreasing pressure.

Fig. \ref{fig:d20fewer} contains photographs of the bending angle for a DS20 circular actuator for both increasing ($P\uparrow$) and decreasing ($P\downarrow$) pressure increments. The actuator had a lower bending angle (a more negative $\psi$) for the same pressure. For this DS20 actuator, all of the photographs from this test are in Appendix \ref{appendix:d20all}.

\begin{figure}[!ht]
    \centering
    \includegraphics[width=6.5 in]{images9/d20fewer.pdf}
    \caption{Samples of photographs used to measure bending angle of a DS20 circular actuator for both increasing and decreasing pressure increments.}
    \label{fig:d20fewer}
\end{figure}

\clearpage
Looking at 70~kPa, the pressure with the largest hysteresis (160$^\circ$), the model predicts $\psi=-70^\circ$. At this pressure, one standard deviation of DS20 actuators had a bending angle of $20\pm60^\circ$ for increasing pressure from 64~kPa and $-140\pm30$ for decreasing pressure from 84~kPa. The uncertainty is partially due to the fact that 7~kPa could not have been small enough to fully capture the behavior. DS20 actuators were the most sensitive to changes in pressure and the actuators did not hold the $\psi=0$ bending angle as well as the stiffer materials. Fig.~\ref{fig:d20at70kpa} contains photographs of DS20 actuators at 70~kPa for both increasing an decreasing pressure, showing the range of $\psi$ at this pressure.

\begin{figure}[!ht]
    \centering
     \includegraphics[width=6.5 in]{images9/d20at70kpa.pdf}
    \caption{Photographs of DS20 actuators at 70~kPa, highlighting the range of $\psi$}
    \label{fig:d20at70kpa}
\end{figure}

The analytical model for bending behavior for DS20 actuators has an RMS error of 70$^\circ$ between the experimental data using both increasing and decreasing pressure. One reason for this error is that the model assumes constant curvature (axial deformation) and constant circumferential deformation along the length of the actuator. The silicone closer to the end caps had less circumferential expansion than the silicone in the center. The non-uniform circumferential deformation induced a non-uniform axial deformation, resulting in non-constant curvature. The non-constant curvature can be seen at 70~kPa (Fig.~\ref{fig:d20at70kpa}), but it is most evident at the higher pressure of 105 kPa. At 105~kPa, with increasing pressure, experimental data shows $\psi=200\pm10^\circ$. Fig.~\ref{fig:d20at105kpa} contains photographs of DS20 actuators at 105~kPa. Near the ends, the actuator has less curvature than near the middle, as marked by the black arrow. Also, note the larger circumferential expansion in the center and how it decreases approaching the ends. 

\begin{figure}[!ht]
    \centering
     \includegraphics[width=4 in]{images9/d20at105kpa.pdf}
    \caption{Photographs of DS20 actuators at 105~kPa. The orange dashed line represents the curvature at the center of the actuator and the black arrow marks the non-uniform curvature at the ends.}
    \label{fig:d20at105kpa}
\end{figure}

\subsection{DS20 and Plastic Deformation}

We found that increasing the pressure beyond 112 kPa caused permanent, plastic deformation in the DS20 silicone. Therefore, to avoid adding more uncertainty in the bending angle from the material properties altering due to the plastic deformation, we did not use higher pressures in DS20 actuators. 

At a pressure of 137 kPa, the actuator approached a bending angle of $-360^\circ$. The substantial circumferential expansion highlights the problem of splitting the insert in the center of the actuator, the white markings highlight damage to the inner silicone wall from the Teflon tape we used during early fabrication methods. Fig. \ref{fig:toomuchpressure} contains photographs of early DS20 actuators at 137~kPa.  
\\
\begin{figure}[!ht]
    \centering
     \includegraphics[width=5 in]{images9/toomuchpressure.jpg}
    \caption{Photographs of early DS20 actuators at 137~kPa.}
    \label{fig:toomuchpressure}
\end{figure}

At $210$~kPa, the semi-circular wall became incredibly thin, and the actuator formed a coil shape like a spring. After depressurizing, we found significant plastic deformation: the actuator returned to a initial bending angle of around $120^\circ$ compared to the original $\psi_0>200^\circ$, deeming it unusable for further testing. 

\clearpage
\section{DS30}

DS30 actuators had a bending range of 424$^\circ$ in 175~kPa. On average, the DS30 actuators had an initial bending angle, $\psi_0$ of 214$^\circ$, and a maximum negative bending angle of $\psi=-210^\circ$. With increasing pressure, the DS30 actuators crossed $\psi=0^\circ$ between 100-114~kPa and 87-104~kPa with decreasing pressure. The overall shape and concavity of the experimental results match those from DS20 but extend along the pressure axis to cross $\psi=0^\circ$ at a higher pressure. 

Similar to DS20, due to the inherent uncertainty in the material models, the analytical model for DS30 underestimates the pressure required for a given bending angle for large positive values of $\psi$ and overestimates the pressure required for large negative values of $\psi$. Unlike DS20, the DS30 model slightly overestimates the $\psi=0^\circ$ crossing pressure, predicting 112~kPa. 

Because DS30 is stiffer than DS20, the bending angle has less hysteresis (only up to 100$^\circ$). Looking at Fig.~\ref{fig:allmaterialvsmodel}, the experimental results have a higher shaded region of pressures for a given bending angle. This difference is not higher uncertainty, but because the DS30 actuators were less sensitive to 7~kPa increments, we could capture more data on how the bending angle varies with pressure. For example, with increasing pressure increments, for DS20, at 70~kPa, we measured $\psi=10\pm70^\circ$, and for DS30 at 106~kPa, we also measured $\psi=10\pm70^\circ$. The uncertainty around $\psi=0^\circ$ is the same for DS20 and DS30. However, at more negative bending angles, DS30 has slightly higher uncertainty. For DS20 at 90~kPa, we measured $\psi=-200\pm30^\circ$ and for DS30 at 154~kPa we measured $\psi=-200\pm40^\circ$. 

\clearpage
Fig. \ref{fig:d30fewer} contains samples photographs used to calculate the bending angle for a DS30 actuator with both increasing and decreasing pressure increments. Photographs of every pressure increment are in Appendix \ref{appendix:d30all}. Note that because DS30 is stiffer than DS20, the 7~kPa increment allowed us to capture more images near $\psi=0^\circ$. 

\begin{figure}[!ht]
    \centering
     \includegraphics[width=6.5 in]{images9/d30fewer.pdf}
    \caption{Samples of photographs of a DS30 actuator used to measure bending angle with both increasing and decreasing pressure increments.}
    \label{fig:d30fewer}
\end{figure}

We observed the same non-constant curvature phenomenon in the DS30 as we did in the DS20 (Fig. \ref{fig:d20at105kpa}). Although slightly less than DS20, the circumferential expansion still induces non-constant curvature, which more significantly affects the bending angle for decreasing pressure than for increasing pressure due to the hyperelastic properties of the silicone material, inducing hysteresis. With decreasing pressure, the actuator is more likely to display non-constant curvature effects because the silicone does not shrink at the same rate it expands. 

\clearpage
\subsection{DS30 with Multiple Concavities}

One of the DS30 actuators we fabricated when pressurized, had an extreme case of non-constant curvature. We will call this varying concavity the ``Seahorse Effect''. This effect is not necessarily unique to DS30; it could occur for any circular actuator. However, since the most dramatic victim of this effect was a DS30 actuator, we include it here. This effect was an unexpected discovery, and it could lead to future exploration of varying the cross-section of circular actuators and how that variation induces non-constant curvature. 

Suppose the circular actuator is fabricated with a non-constant cross-section; either there was a misshaped insert, or the silicone was damaged, resulting in a thinner $t$ and larger $a$ at some point along the actuator. Areas with thinner semi-circular walls have larger circumferential expansion with increasing pressure. The uneven circumferential deformation, if significantly more than a circular actuator with a uniform cross-section, induces non-constant curvature along the actuator, which leads to the ``Seahorse Effect'': an actuator shaped like a seahorse. Fig. \ref{fig:seahorsefewer} contains photographs comparing a DS30 actuator with and without this defect. Appendix \ref{appendix:d30seahorse} contains photographs at all pressures of this DS30 actuator. 

\begin{figure}[!ht]
    \centering
     \includegraphics[width=6.5 in]{images9/seahorsefewer.pdf}
    \caption{Photographs of DS30 actuators with and without experiencing the ``Seahorse Effect''. The orange dashed circles indicate the curvature at the point of the most circumferential expansion, and the arrow points to the undesired circumferential expansion.}
    \label{fig:seahorsefewer}
\end{figure}

Comparing the bending behavior of the two actuators in Fig. \ref{fig:seahorsefewer}, at 98 kPa, the black arrow points to the additional circumferential deformation, which leads to the non-constant curvature of the fiberglass fabric marked with the black line. At 112~kPa, the orange dashed lines indicate the primary curvature of the actuator. For the actuator in the bottom row, the end on the left experiences curvature not only away from the orange circle but also has the opposite concavity. At 154~kPa, two problems are occurring: the ends have broken away from the curvature of the orange dashed circle, and for the actuator with the ``Seahorse Effect'', the primary curvature is not in the center of the actuator, more of the end on the left is away from the circle than on the right. 

Another way of visualizing the effect is by moving the point of maximum circumferential expansion away from the center of the actuator as indicated at 98 kPa with the black arrow in Fig.~\ref{fig:seahorsefewer}. At pressures where we expect $\psi\thickapprox0^\circ$, the actuator exhibits both positive and negative concavity. The average curvature measured using OpenCV was close to $\psi=0^\circ$, but the actual shape of the actuator was not straight, a flaw in our method of measuring bending angle. We included these results in Fig. \ref{fig:allmaterialvsmodel}, which increases the uncertainty of the bending angle for DS30 at higher pressures. 

\clearpage
\section{SS40}

Within the limitations of the pressure rig (a max pressure of 256~kPa), circular actuators fabricated from SS40 silicone reached $\psi=0^\circ$ and can display small negative bending angles, a total range of 265$^\circ$. 
The SS40 bending angle results further show the non-linearity of the shore hardness scale in relation to the behavior of the circular actuators. Between DS20 and DS30, pressure of the $\psi=0^\circ$ cross-over point increased by about 40~kPa. From DS30 to SS40, we see an increase of about 120~kPa. 

SS40 actuators had visibly closer to constant curvature near $\psi=0^\circ$, a substantial improvement compared to DS20 and DS30 actuators. Because the material is stiffer, less circumferential deformation leads to less variation in axial deformation. Fig. \ref{fig:ss40fewer} contains some of the photographs used to measure the bending angle for an SS40 actuator. This particular actuator crossed $\psi=0^\circ$ between the photographs taken at 189 and 217~kPa. The lowest bending angle achieved is -37$^\circ$. Appendix \ref{appendix:s40all} contains all the photographs this test of this actuator. 

\begin{figure}[!ht]
    \centering
     \includegraphics[width=6.5 in]{images9/s40fewer.pdf}
    \caption{Photographs of a SS40 actuator used to calculate bending angle for both increasing and decreasing pressure increments.}
    \label{fig:ss40fewer}
\end{figure}

\clearpage
Similar to DS20 and DS30, the analytical model for SS40 underestimates the pressure required to achieve a bending angle for positive $\psi$ values. As seen in Fig. \ref{fig:allmaterialvsmodel}, the model begins overestimating the bending angle around 160~kPa. As expected from a stiffer material, the SS40 actuators display less hysteresis in bending angle (only up to 30 $^circ$). For SS40, the model has an RMS error of 30$^\circ$, less than half the error for softer silicones. For $\psi=0^\circ$, the model predicts 235 kPa. Experimental data at 231 kPa shows $\psi=-10\pm10^\circ$ for increasing pressure and $\psi=-20\pm20^\circ$ for decreasing pressure. The model also predicts $\psi=-\psi_0$ at 378~kPa, but we could not verify this within the limitations of the pressurization equipment. 

\clearpage
\section{SS50}

The stiffest of the four materials, in 256~kPa, the SS50 actuators displayed a bending range of just 64$^\circ$. Due to the small bending range and hysteresis (up to 3$^\circ$) of the stiff material, the RMS error of the analytical model compared to experimental data was 4.5$^\circ$. Once again, we expect a significant increase in $\psi=0^\circ$ compared to the jump between DS30 and SS40; the model predicts $\psi=0$ at 674~kPa (97 psi). Fig \ref{fig:s50fewer} contains samples of photographs of an SS50 actuator used to calculate bending angle. All of the photographs of this actuator during this test are in Appendix \ref{appendix:s50all}. 
\\
\begin{figure}[ht]
    \centering
     \includegraphics[width=6.5 in]{images9/s50fewer.pdf}
    \caption{Photographs of a SS50 actuator with increasing and decreasing pressure increments.}
    \label{fig:s50fewer}
\end{figure}
